\section{Le Modèle Standard et ses limitations}
\label{sec:ms}

Le Modèle Standard (MS) est une théorie des intéractions fondamentales
décrivant la nature avec un niveaux de précision impréssionant. Ce
succès a culminé en 2012 lorsque les expérience ATLAS et CMS ont
observé le boson de Higgs en laboratoire pour une première
fois. C'étais alors la seule grande prédiction du MS encore non
confirmée. 

La structure théorique du modèle est présentée dans la
section~\ref{sec:ms:th}, et quelques grand problèmes non résolus dans
le MS sont exposés dans la section~\ref{sec:ms:problemes}.

\subsection{Survol théorique}
\label{sec:ms:th}

Concrètement, le Modèle Standard décrit les particules fondamentales
de la nature ainsi que les forces par lesquelles elles
intéragissent. Formellement, les particules
(section~\ref{sec:ms:th:particules}) sont vues comme étant des
excitations localisées de de différents champs dont les intéractions
sont décrites par les différents secteurs du Lagrangien du MS
(section~\ref{sec:ms:th:struct}).

\subsubsection{Les particules du Modèle Standard}
\label{sec:ms:th:particules}

On dénote d'abord deux grands types de particles, défini par la nature
de leur spin. Les particles ayant spin entier sont appelés
\emph{bosons}, tandis que les particules avec spin demi-entier sont
appelés \emph{fermions}. Les fermions sont ensuite séparés en deux
sous-classes, selon leur couleur: les particules non-colorées sont
appelées \emph{leptons} et celles ayant une couleur, les
\emph{quarks}. La couleur est l'analogue chromodynamique de la charge
électrique.  Les différentes particules du MS ainsi que leurs
propriétés sont dans la table~\ref{tab:ms_particules}.

Une des caractéristiques frapantes du Modèle Standard est qu'il existe
une \emph{anti-particule} associée à chaque particule. Si la particule
est chargée, son anti-particule a la charge opposée. Le gluon, le
photon et le Z sont leurs propres anti-particules.

\begin{table}[h!]
  \centering
  \begin{tabular}{|c|c|c|c|c|c|}
  \hline
  symbole   & nom                 & spin & charge & coloré & Masse \\ \hline
  $\gamma$/$A_\mu$  & photon              & 1    & 0     & non    &     \\ \hline
  $g$       & gluon               & 1    & 0     & oui    &     \\ \hline
  $Z$       & boson Z             & 1    & 0     & non    & 91.1876 $\pm$ 0.0021 GeV \cite{olive_review_2014} \\ \hline
  $W^+/W^-$ & bosons W            & 1    & +1/-1 & non    & 80.385 $\pm$ 0.015 GeV \cite{olive_review_2014} \\ \hline
  $h$       & Higgs               & 0    & 0     & non    & 125.09 $\pm$ 0.21 $\pm$ 0.11 GeV \cite{atlas_collaboration_combined_2015}  \\ \hline
  $e^-$     & \'electron          & 1/2  & -1    & non    & 0.510998928 $\pm$ 0.000000011 \cite{mohr_codata_2012} \\ \hline
  $\nu_e$   & \'electron-neutrino & 1/2  & 0     & non    & $\bar{\nu}$: < 2 eV \cite{olive_review_2014} / $\nu$: < 460 eV \cite{yasumi_mass_1994} \\ \hline
  $\mu^-$   & muon                & 1/2  & -1    & non    & 105.6583715 $\pm$ 0.0000035 MeV \cite{mohr_codata_2012}    \\ \hline
  $\nu_\mu$ & muon-neutrino       & 1/2  & 0     & non    & < 0.19 MeV \cite{olive_review_2014}    \\ \hline
  $\tau^-$  & tau                 & 1/2  & -1    & non    & 1.77686 $\pm$ 0.00012 GeV \cite{olive_review_2014}    \\ \hline
  $\nu_\tau$ & tau-neutrino       & 1/2  & 0     & non    &  < 18.2 MeV \cite{al_upper_1998}   \\ \hline
  $d$       & down                & 1/2  & -1/3  & oui    & 4.8 + 0.5 - 0.3 MeV \cite{olive_review_2014}     \\ \hline
  $u$       & up                  & 1/2  & 2/3   & oui    & 2.3 + 0.7 - 0.5 MeV \cite{olive_review_2014}    \\ \hline
  $s$       & strange             & 1/2  & -1/3  & oui    & 95 $\pm$ 5 MeV \cite{olive_review_2014}    \\ \hline
  $c$       & charm               & 1/2  & 2/3   & oui    & 1.275 $\pm$ 0.025 GeV \cite{olive_review_2014}    \\ \hline
  $b$       & bottom              & 1/2  & -1/3  & oui    & 4.18 $\pm$ 0.03 GeV \cite{olive_review_2014}    \\ \hline
  $t$       & top                 & 1/2  & 2/3   & oui    & 173.21 $\pm$ 0.51 $\pm$ 0.71 GeV \cite{olive_review_2014}    \\ \hline
\end{tabular}
\caption{Les particules du Modèle Standard et leurs propriétés.}
\label{tab:ms_particules}
\end{table}

\subsubsection{Structure du Modèle Standard}
\label{sec:ms:th:struct}

% Intro

% EWK
Le modèle électrofaible prédit l'existance de quatre champs de jauge:
$W^{(1)}_\mu$, $W^{(2)}_\mu$, $W^{(3)}_\mu$ et $B_\mu$. La théorie
prédit deux nombres quantiques pour chacun de ces bosons: l'isopsin
faible, $I^{(3)}_W$, et l'hypercharge, $Y$. Les bosons W et Z, ainsi
que le photon, sont des combinaisons linéaires de ces bosons:
\begin{eqnarray}
  \label{eq:ewk_mix}
  W^{\pm}_\mu = \frac{1}{\sqrt{2}}(W^{(1)}_\mu \mp W^{(2)}_\mu)  \\
  Z_\mu = -B_\mu\ sin\ \theta_W + W^{(3)}_\mu cos\ \theta_W \\
  A_\mu = B_\mu\ cos\ \theta_W + W^{(3)}_\mu sin\ \theta_W
\end{eqnarray}
L'angle de mélange électrofaible, $\theta_W$, est un paramètre du
modèle. Les nombres quantiques $I^{(3)}_W$ et $Y$ déterminent la
charge électrique à travers la relation
\begin{eqnarray}
  Q = \frac{Y}{2} + I^{(3)}_W
\end{eqnarray}
La constante de couplage électrique, $e$, est reliée à la constante de
couplage faible, $g_W$, à travers l'angle de mélange:
\begin{eqnarray}
  e = g_W\ sin\ \theta_W
\end{eqnarray}

Le boson $Z$ et le photon sont les médiateurs de la force
électrofaible neutre. Les intéractions dans ce secteurs conservent la
charge et les nombres leptoniques et baryoniques. Par conséquant, les
couplages possibles sont entre le boson et un paire
fermion-antifermion de même saveur.

Les bosons $W$ sont chargés et conséquamment, la valeur absolue de la
charge des deux particules au sommet doivent être
différente. L'intéraction faible chargé est la plus spéciales du MS en
ce sens qu'elle viole la conservation de saveur: les leptons couplent
avec leur neutrinos respectifs et les quarks sont couplés les uns aux
autres, la force des couplages étant donnés par la matrice CKM.

% Higgs
La théorie électrofaible est complété par le champs de Higgs, qui
permet des couplages entre le boson de Higgs et des paires de bosons $W$ ou $Z$
ou fermions-antifermions. Ce champ est relié au mécanisme de Higgs,
par lequel les bosons de cette théorie acquièrent leurs masses. \\

% QCD
Le Modèle Standard est complété par le secteur chromodynamique, qui
décrit l'intéraction entre particules portant une couleur. Le
médiateur de cette force est le gluon, et il existe seulement des couplages
qqg, ggg et gggg puisque les leptons et les autres
bosons ne sont pas colorés. 

Une caractéristique de la chromodynamique est que la constante de
couplage $\alpha_s$ décroit lorsque l'échelle d'énergie augmente: ce
phénomène est appellé \emph{liberté asymptotique}. La
figure~\ref{fig:alpha_s} montre un sommaire des principales mesures
d'$\alpha_s$ à diverse échelles d'énergie, confirmant solidement cette
particularité de la théorie. 

\begin{figure}
  \centering
  \includegraphics{alpha_s.pdf}
  \caption{Sommaire des principales mesures de la constante de couplage forte, $\alpha_s$, à différentes échelles d'énergies ($Q$). Figure tirée de la réf. \cite{olive_review_2014}}
\label{fig:alpha_s}
\end{figure}

Une conséquence importe de ce phénomène est que les quarks sont
confinés dans des états liés, puisque la constante de couplage
augmente rapidement alors que l'échelle de distance augmente. Pour
produire des intéractions chromidynamiques en laboratoire, il faut
donc collisionner des états liés de quarks et gluons, communément
appelés \emph{hadrons}. Or, ceci complique les calculs, puisqu'on ne
sait pas entre quels constituants une intéraction donnée a eu lieu. Il
faut donc exprimer les sections efficaces en termes de fonctions de
structures qui paramétrisent la forme spatiale et le contenue en
parton~\footnote{parton $\equiv$ quark, anti-quark ou gluon} d'un
hadron donné. Ces fonctions ne sont en général pas calculables
théoriquement. Elles sont plutot exprimées en terme des fonctions de
densités de partons (PDF) $f_q(x,Q^2)$ calculées expérimentalement et
exprimée en fonction de la fraction de l'impulsion du hadron porté par
le parton $q$ et l'échelle d'énergie du processus $Q^2$. Une
représentation graphique de la PDF du proton
\texttt{NNPDF2.3}~\cite{ball_parton_2013} est montrée dans la
figure~\ref{fig:pdf}.

\begin{figure}
  \centering
  \includegraphics{nnpdf23.pdf}
  \caption{Représentation graphique de la PDF du proton
    \texttt{NNPDF2.3}~\cite{ball_parton_2013}, à deux échelles
    d'énergies ($\mu$) différentes et exprimé en fonction de la
    fraction d'impulsion porté par un parton donné, $x$. Figure tirée de la
    réf. \cite{olive_review_2014}.}
  \label{fig:pdf}
\end{figure}

Les intéractions à hautes énergies peuvent être calculés
perturbativement. Ces intéractions peuvent avoir des partons libres
dans l'état final. Or, la liberté asymptotique veut qu'ils soient
confinés en états liés; il y a donc une phase successive
d'\emph{hadronisation}, lors de laquelle les quarks irradient des
gluons et les gluons se désintègrent en pairs quarks-antiquarks, de
façon à créer une cascade de hadrons dans l'état final. Les hadrons
d'une même cascade tendent à être collimés dans une même direction et
forment donc un amat d'énergie dans les calorimètres appellés
\emph{gerbes}~\footnote{Traduction du terme anglais «jet»}.

\subsection{Validation expérimentale}
\label{sec:ms:exp}

\subsubsection{Mesures du secteur électrofaible}
\label{sec:ms:exp:ewk}

\subsubsection{Mesures du secteur chromodynamique}
\label{sec:ms:exp:qcd}

\subsubsection{Mesures du secteur Higgs}
\label{sec:ms:exp:higgs}

\subsection{Problèmes avec le Modèle Standard}
\label{sec:ms:problemes}

%%% Local Variables:
%%% mode: latex
%%% TeX-master: "memoire"
%%% End:
