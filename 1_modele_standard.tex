\section{Le Modèle Standard et ses limitations}
\label{sec:ms}

Le Modèle Standard (MS) est une théorie des intéractions fondamentales
décrivant la nature avec un niveaux de précision impréssionant. Ce
succès a culminé en 2012 lorsque les expérience ATLAS et CMS ont
observé le boson de Higgs en laboratoire pour une première
fois. C'étais alors la seule grande prédiction du MS encore non
confirmée. 

La structure théorique du modèle est présentée dans la
section~\ref{sec:ms:th}, et quelques grand problèmes non résolus dans
le MS sont exposés dans la section~\ref{sec:ms:problemes}.

\subsection{Survol théorique}
\label{sec:ms:th}

Concrètement, le Modèle Standard décrit les particules fondamentales
de la nature ainsi que les forces par lesquelles elles
intéragissent. Formellement, les particules
(section~\ref{sec:ms:th:particules}) sont vues comme étant des
excitations localisées de de différents champs dont les intéractions
sont décrites par les différents secteurs du Lagrangien du MS
(section~\ref{sec:ms:th:struct}).

\subsubsection{Les particules du Modèle Standard}
\label{sec:ms:th:particules}

On dénote d'abord deux grands types de particles, défini par la nature
de leur spin. Les particles ayant spin entier sont appelés
\emph{bosons}, tandis que les particules avec spin demi-entier sont
appelés \emph{fermions}. Les fermions sont ensuite séparés en deux
sous-classes, selon leur couleur: les particules non-colorées sont
appelées \emph{leptons} et celles ayant une couleur, les
\emph{quarks}. La couleur est l'analogue chromodynamique de la charge
électrique.  Les différentes particules du MS ainsi que leurs
propriétés sont dans la table~\ref{tab:ms_particules}.

Une des caractéristiques frapantes du Modèle Standard est qu'il existe
une \emph{anti-particule} associée à chaque particule. Si la particule
est chargée, son anti-particule a la charge opposée. Le gluon, le
photon et le Z sont leurs propres anti-particules.

\begin{table}[h!]
  \centering
  \begin{tabular}{|c|c|c|c|c|c|}
  \hline
  symbole   & nom                 & spin & charge & coloré & Masse \\ \hline
  $\gamma$  & photon              & 1    & 0     & non    &     \\ \hline
  $g$       & gluon               & 1    & 0     & oui    &     \\ \hline
  $Z$       & boson Z             & 1    & 0     & non    & 91.1876 $\pm$ 0.0021 GeV \cite{olive_review_2014} \\ \hline
  $W^+/W^-$ & bosons W            & 1    & +1/-1 & non    & 80.385 $\pm$ 0.015 GeV \cite{olive_review_2014} \\ \hline
  $h$       & Higgs               & 0    & 0     & non    & 125.09 $\pm$ 0.21 $\pm$ 0.11 GeV \cite{atlas_collaboration_combined_2015}  \\ \hline
  $e^-$     & \'electron          & 1/2  & -1    & non    & 0.510998928 $\pm$ 0.000000011 \cite{mohr_codata_2012} \\ \hline
  $\nu_e$   & \'electron-neutrino & 1/2  & 0     & non    & $\bar{\nu}$: < 2 eV \cite{olive_review_2014} / $\nu$: < 460 eV \cite{yasumi_mass_1994} \\ \hline
  $\mu^-$   & muon                & 1/2  & -1    & non    & 105.6583715 $\pm$ 0.0000035 MeV \cite{mohr_codata_2012}    \\ \hline
  $\nu_\mu$ & muon-neutrino       & 1/2  & 0     & non    & < 0.19 MeV \cite{olive_review_2014}    \\ \hline
  $\tau^-$  & tau                 & 1/2  & -1    & non    & 1.77686 $\pm$ 0.00012 GeV \cite{olive_review_2014}    \\ \hline
  $\nu_\tau$ & tau-neutrino       & 1/2  & 0     & non    &     \\ \hline
  $d$       & down                & 1/2  & -1/3  & oui    & 4.8 + 0.5 - 0.3 MeV \cite{olive_review_2014}     \\ \hline
  $u$       & up                  & 1/2  & 2/3   & oui    & 2.3 + 0.7 - 0.5 MeV \cite{olive_review_2014}    \\ \hline
  $s$       & strange             & 1/2  & -1/3  & oui    & 95 $\pm$ 5 MeV \cite{olive_review_2014}    \\ \hline
  $c$       & charm               & 1/2  & 2/3   & oui    & 1.275 $\pm$ 0.025 GeV \cite{olive_review_2014}    \\ \hline
  $b$       & bottom              & 1/2  & -1/3  & oui    & 4.18 $\pm$ 0.03 GeV \cite{olive_review_2014}    \\ \hline
  $t$       & top                 & 1/2  & 2/3   & oui    & 173.21 $\pm$ 0.51 $\pm$ 0.71 GeV \cite{olive_review_2014}    \\ \hline
\end{tabular}
\caption{Les particules du Modèle Standard et leurs propriétés.}
\label{tab:ms_particules}
\end{table}

\subsubsection{Structure du Modèle Standard}
\label{sec:ms:th:struct}

\blindtext

\subsection{Validation expérimentale}
\label{sec:ms:exp}

\subsubsection{Mesures du secteur électrofaible}
\label{sec:ms:exp:ewk}

\subsubsection{Mesures du secteur chromodynamique}
\label{sec:ms:exp:qcd}

\subsubsection{Mesures du secteur Higgs}
\label{sec:ms:exp:higgs}

\subsection{Problèmes avec le Modèle Standard}
\label{sec:ms:problemes}
