\section{Le Modèle Standard et ses limitations}
\label{sec:ms}

Le Modèle Standard (MS) est une théorie des intéractions fondamentales
décrivant la nature avec un niveaux de précision impréssionant. Ce
succès a culminé en 2012 lorsque les expérience ATLAS et CMS ont
observé le boson de Higgs en laboratoire pour une première
fois. C'étais alors la seule grande prédiction du MS encore non
confirmée. 

La structure théorique du modèle est présentée dans la
section~\ref{sec:ms:th}, et quelques grand problèmes non résolus dans
le MS sont exposés dans la section~\ref{sec:ms:problemes}.

\subsection{Survol théorique}
\label{sec:ms:th}

Concrètement, le Modèle Standard décrit les particules fondamentales
de la nature ainsi que les forces par lesquelles elles
intéragissent. Formellement, les particules
(section~\ref{sec:ms:th:particules}) sont vues comme étant des
excitations localisées de de différents champs dont les intéractions
sont décrites par les différents secteurs du Lagrangien du MS
(section~\ref{sec:ms:th:struct}).

\subsubsection{Les particules du Modèle Standard}
\label{sec:ms:th:particules}

On dénote d'abord deux grands types de particles, défini par la nature
de leur spin. Les particles ayant spin entier sont appelés
\emph{bosons}, tandis que les particules avec spin demi-entier sont
appelés \emph{fermions}.

Les particules sont ensuite défines selon leurs charge électriques et
leurs masses. La masse de la plupart des particles du MS sont des
paramètres du modèle et sont donc définies expérimentalement. Les
différentes particules du MS ainsi que leurs propriétés sont dans la table.

\begin{table}[h!]
  \centering
  \begin{tabular}{|c|c|c|c|c|c|}
  \hline
  symbole   & nom      & spin & charge & coloré & Masse \\ \hline
  $\gamma$  & photon   & 1    & 0      & non    &     \\ \hline
  $g$       & gluon    & 1    & 0      & oui    &     \\ \hline
  $Z$       & boson Z  & 1    & 0      & non    & 91.1876 $\pm$ 0.0021 GeV \cite{olive_review_2014} \\ \hline
  $W^+/W^-$ & bosons W & 1    & +1/-1  & non    & 80.385 $\pm$ 0.015 GeV \cite{olive_review_2014} \\ \hline
  $h$       & Higgs    & 0    & 0      & non    & 125.09 $\pm$ 0.21 $\pm$ 0.11 GeV \cite{atlas_collaboration_combined_2015}     \\ \hline
\end{tabular}
\caption{Bosons du Modèle Standard}
\end{table}
\subsubsection{Structure du Modèle Standard}
\label{sec:ms:th:struct}

\subsection{Validation expérimentale}
\label{sec:ms:exp}

\subsubsection{Mesures du secteur électrofaible}
\label{sec:ms:exp:ewk}

\subsubsection{Mesures du secteur chromodynamique}
\label{sec:ms:exp:qcd}

\subsubsection{Mesures du secteur Higgs}
\label{sec:ms:exp:higgs}

\subsection{Problèmes avec le Modèle Standard}
\label{sec:ms:problemes}
