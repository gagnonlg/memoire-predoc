\chapter{Le Modèle Standard et ses limitations}
\label{sec:ms}

Le Modèle Standard est une théorie des interactions fondamentales
décrivant la nature avec un niveaux de précision impressionnant.  La
structure théorique du modèle est présentée dans la
section~\ref{sec:ms:th}, basée en majeure partie sur les références
\cite{olive_ewk_2014}, \cite{olive_qcd_2014}
et~\cite{thomson_modern_2013}. Quelques grand problèmes non résolus
sont ensuite exposés dans la section~\ref{sec:ms:problemes}.

\section{Survol théorique}
\label{sec:ms:th}

Le Modèle Standard décrit les particules fondamentales de la nature
ainsi que les forces par lesquelles elles interagissent. Formellement,
les particules (section~\ref{sec:ms:th:particules}) sont vues comme
étant des excitations localisées de de différents champs dont les
interactions sont décrites par les différents secteurs
du modèle (section~\ref{sec:ms:th:struct}).

\subsection{Les particules du Modèle Standard}
\label{sec:ms:th:particules}

Il existe deux grands types de particles, défini par la nature
de leur spin. Les particles ayant spin entier sont appelés
\emph{bosons}, tandis que les particules avec spin demi-entier sont
appelés \emph{fermions}. Les fermions sont ensuite séparés en deux
sous-classes, selon leur couleur~\footnote{l'analogue chromodynamique
  de la charge électrique}: les fermions non-colorées sont appelées
\emph{leptons} et ceux ayant une couleur, les \emph{quarks}. 

Le modèle standard compte six leptons. L'électron ($e^-$), le muon
($\mu^-$) et le tau ($\tau^-$) sont chargés négativement, tandis que
les trois neutrinos ($\nu_e$, $\nu_\mu$ et $\nu_\tau$) sont
électriquement neutres. Il y a aussi six quarks, portant tous une
charge fractionaire. le \emph{down} ($d$), le \emph{strange} ($s$) et
le \emph{bottom} ($b$) ont une charge de $-\frac{1}{3}$ alors que le
\emph{up}, le \emph{charm} ($c$) et le \emph{top} ($t$) ont une charge
de $\frac{2}{3}$.

Les masses des
leptons et des quarks sont ordonnées de la façon suivante:
\begin{eqnarray}
  m_e < m_\mu < m_\tau \nonumber \\
  m_d < m_s < m_b \\
  m_u < m_c < m_t \nonumber 
\end{eqnarray}

On dit donc qu'il y a trois générations de fermions, selon leur
échelle de masse. On associe aussi une génération à chaque neutrino,
selon qu'il sont électroniques (première génération), muoniques
(deuxième génération) ou tauoniques (troisième génération).

% Les
% propriétés des différentes particules du MS sont résumées dans la
% table~\ref{tab:ms_particules}.

En addition aux fermions, il y a cinq bosons. Le photon ($\gamma$),
neutre et sans masse, le $Z$, neutre et massif, le $W$, chargé et
massif ainsi que le gluon ($g$), neutre, sans masse et coloré ont tous
un spin de 1 (bosons vectoriels) tandis que le boson de Higgs ($h$),
neutre et massif, a un spin nul (boson scalaire).

Une des caractéristiques frappantes du Modèle Standard est qu'il existe
une \emph{anti-particule} associée à chaque particule. Si la particule
est chargée, son anti-particule a la charge opposée. Le gluon, le
photon et le Z sont leurs propres anti-particules.

% \begin{table}[h!]
%   \centering
%   \begin{tabular}{|c|c|c|c|c|c|}
%   \hline
%   symbole   & nom                 & spin & charge & coloré & Masse \\ \hline
%   $\gamma$/$A_\mu$  & photon              & 1    & 0     & non    & 0    \\ \hline
%   $g$       & gluon               & 1    & 0     & oui    &  0   \\ \hline
%   $Z$       & boson Z             & 1    & 0     & non    & 91.1876 $\pm$ 0.0021 GeV \cite{olive_review_2014} \\ \hline
%   $W^-$     & bosons W            & 1    & -1    & non    & 80.385 $\pm$ 0.015 GeV \cite{olive_review_2014} \\ \hline
%   $h$       & Higgs               & 0    & 0     & non    & 125.09 $\pm$ 0.21 $\pm$ 0.11 GeV \cite{atlas_collaboration_combined_2015}  \\ \hline
%   $e^-$     & électron          & 1/2  & -1    & non    & 0.510998928 $\pm$ 0.000000011 \cite{mohr_codata_2012} \\ \hline
%   $\mu^-$   & muon                & 1/2  & -1    & non    & 105.6583715 $\pm$ 0.0000035 MeV \cite{mohr_codata_2012}    \\ \hline
%   $\tau^-$  & tau                 & 1/2  & -1    & non    & 1.77686 $\pm$ 0.00012 GeV \cite{olive_review_2014}    \\ \hline
%   $\nu_e$   & électron-neutrino & 1/2  & 0     & non    & $\bar{\nu}$: < 2 eV \cite{olive_review_2014} / $\nu$: < 460 eV \cite{yasumi_mass_1994} \\ \hline
%   $\nu_\mu$ & muon-neutrino       & 1/2  & 0     & non    & < 0.19 MeV \cite{olive_review_2014}    \\ \hline
%   $\nu_\tau$ & tau-neutrino       & 1/2  & 0     & non    &  < 18.2 MeV \cite{al_upper_1998}   \\ \hline
%   $d$       & quark down          & 1/2  & -1/3  & oui    & 4.8 + 0.5 - 0.3 MeV \cite{olive_review_2014}     \\ \hline
%   $u$       & quark up            & 1/2  & 2/3   & oui    & 2.3 + 0.7 - 0.5 MeV \cite{olive_review_2014}    \\ \hline
%   $s$       & quark strange       & 1/2  & -1/3  & oui    & 95 $\pm$ 5 MeV \cite{olive_review_2014}    \\ \hline
%   $c$       & quark charm         & 1/2  & 2/3   & oui    & 1.275 $\pm$ 0.025 GeV \cite{olive_review_2014}    \\ \hline
%   $b$       & quark bottom        & 1/2  & -1/3  & oui    & 4.18 $\pm$ 0.03 GeV \cite{olive_review_2014}    \\ \hline
%   $t$       & quark top           & 1/2  & 2/3   & oui    & 173.21 $\pm$ 0.51 $\pm$ 0.71 GeV \cite{olive_review_2014}    \\ \hline
% \end{tabular}
% \caption{Les particules du Modèle Standard et leurs propriétés. Chaque particule a une anti-particule associée,
%   ayant une charge électrique inverse. Le gluon, le photon, le Z et potentiellement les neutrinos sont leurs
%   propres anti-particules. }
% \label{tab:ms_particules}
% \end{table}


\subsection{Structure du Modèle Standard}
\label{sec:ms:th:struct}

% Intro
Le Modèle Standard peut être séparé en trois secteurs principaux: le
secteur électrofaible, le secteur Higgs et le secteur chromodynamique.

\subsubsection{Secteur électrofaible}
\label{sec:ms:th:struct:ewk}

% EWK
Les forces faibles et électromagnétiques sont comprises comme étant
deux aspects d'une seule et même force, la force électrofaible, depuis
les travaux de Glashow, Salam et Weinberg dans les années 1960. Le
modèle électrofaible prédit l'existence de quatre champs de jauge:
$W^{(1)}_\mu$, $W^{(2)}_\mu$, $W^{(3)}_\mu$ et $B_\mu$. La théorie
prédit deux nombres quantiques pour chacun de ces bosons: l'isopsin
faible, $I^{(3)}_W$, et l'hypercharge, $Y$. Les bosons W et Z, ainsi
que le photon, sont des combinaisons linéaires de ces états et la
charge électrique est une combinaison  de $Y$ et $I^{(3)}_W$.

% \begin{eqnarray}
%   \label{eq:ewk_mix}
%   W^{\pm}_\mu = \frac{1}{\sqrt{2}}(W^{(1)}_\mu \mp W^{(2)}_\mu)  \\
%   Z_\mu = -B_\mu\ sin\ \theta_W + W^{(3)}_\mu cos\ \theta_W \\
%   A_\mu = B_\mu\ cos\ \theta_W + W^{(3)}_\mu sin\ \theta_W
% \end{eqnarray}

% L'angle de mélange électrofaible, $\theta_W$, est un paramètre du
% modèle. Les nombres quantiques $I^{(3)}_W$ et $Y$ déterminent la
% charge électrique à travers la relation
% \begin{eqnarray}
%   Q = \frac{Y}{2} + I^{(3)}_W
% \end{eqnarray}

La constante de couplage électrique, $e$, est reliée à la constante de
couplage faible, $g_W$, à travers l'angle de mélange $\theta_W$ qui
est un paramètre du modèle: 
\begin{eqnarray}
\label{eq:egw}
  e = g_W\ sin\ \theta_W
\end{eqnarray}

Le boson $Z$ et le photon sont les médiateurs de la force faible
neutre et électromagnétique. Les interactions dans ce secteurs
conservent la charge et les nombres leptoniques et baryoniques. Par
conséquent, les couplages possibles sont entre le boson et un paire
fermion-antifermion de même saveur.

Les bosons $W$ sont chargés et conséquemment, la valeur de la charge
des deux particules au sommet doivent être différente. L'interaction
faible chargé est la plus spéciales du modèle standard en ce sens
qu'elle viole la conservation de saveur: les leptons couplent avec
leur neutrinos respectifs et les quarks de charge -1/3 sont
couplés à ceux de charge 2/3.

\subsubsection{Secteur Higgs}
% Higgs
Un problème avec la théorie électrofaible est que les bosons sont sans
masses, ce qui est en désaccord avec les observations
expérimentales. Le champ de Higgs est donc incorporé au modèle
standard pour expliquer le mécanisme par lequel les bosons
électrofaibles acquièrent leurs masses. Une conséquence expérimentale
importante de ce champ est la présence d'un boson scalaire neutre
massif, le Higgs, qui a été confirmée en 2012 par les expériences
ATLAS~\cite{aad_observation_2012}
et~CMS~\cite{chatrchyan_observation_2012}.

\subsubsection{Secteur chromodynamique}
% QCD
Le Modèle Standard est complété par le secteur chromodynamique, qui
décrit l'interaction entre particules portant une couleur. Le
médiateur de cette force est le gluon, et il existe seulement des
couplages quark-quark-$g$, $ggg$ et $gggg$ puisque les leptons et les
autres bosons ne sont pas colorés. L'interaction chromodynamique est
aussi appelée force forte. Une caractéristique importante de cette
force est que sa constante de couplage, $\alpha_s$, décroît lorsque
l'échelle d'énergie augmente: ce phénomène est appelé \emph{liberté
  asymptotique}. La figure~\ref{fig:alpha_s} montre un sommaire des
principales mesures d'$\alpha_s$ à diverse échelles d'énergie,
confirmant solidement cette particularité de la théorie.

\begin{figure}
  \centering
  \includegraphics{alpha_s.pdf}
  \caption{Sommaire des principales mesures de la constante de couplage forte, $\alpha_s$, à différentes échelles d'énergies ($Q$). Figure tirée de la réf.~\cite{olive_qcd_2014}.}
\label{fig:alpha_s}
\end{figure}

Une conséquence importe de ce phénomène est que les quarks sont
confinés dans des états liés, puisque la constante de couplage
augmente rapidement alors que l'échelle de distance augmente. Pour
produire des interactions fortes en laboratoire, il faut donc
collisionner des états liés de quarks et gluons, communément appelés
\emph{hadrons}. Or, ceci complique les calculs, puisqu'il n'est pas
possible de savoir entre quels constituants une interaction donnée a
eu lieu. Il faut donc exprimer les sections efficaces en termes de
fonctions de structures qui paramètrent le contenue en
parton~\footnote{parton $\equiv$ quark, anti-quark ou gluon} d'un
hadron donné. Ces fonctions ne sont en général pas calculables
théoriquement, donc elles doivent plutôt êtres exprimées en terme des
fonctions de densités de partons (PDF) calculées expérimentalement et
exprimée en fonction de la fraction de l'impulsion du hadron porté par
un parton donné et l'échelle d'énergie du processus. Une
représentation graphique de la PDF du proton
\texttt{NNPDF2.3}~\cite{ball_parton_2013} est montrée dans la
figure~\ref{fig:pdf}, où il est possible de voir que le proton n'est
pas seulement constitué de quarks $u$ et $d$.

\begin{figure}
  \centering
  \includegraphics{nnpdf23.pdf}
  \caption{Représentation graphique de la PDF du proton
    \texttt{NNPDF2.3}~\cite{ball_parton_2013}, à deux échelles
    d'énergies ($\mu$) différentes et exprimé en fonction de la
    fraction d'impulsion porté par un parton donné, $x$. Figure tirée de la
    réf. \cite{olive_qcd_2014}.}
  \label{fig:pdf}
\end{figure}

Les interactions peuvent engendrer des partons libres dans l'état
final. Or, la liberté asymptotique veut qu'ils soient confinés en
états liés; il y a donc une phase successive d'\emph{hadronisation},
lors de laquelle les quarks irradient des gluons et les gluons se
désintègrent en pairs quarks-antiquarks, de façon à créer une cascade
de hadrons dans l'état final (voir
figure~\ref{fig:hadronisation}). Les hadrons d'une même cascade
tendent à être collimés dans une même direction et forment des
dépôts d'énergie dans les calorimètres appelés
\emph{gerbes}~\footnote{Traduction du terme anglais
  «jet»}.

\begin{figure}
  \centering
  \includegraphics{hadronisation.jpg}
  \caption{Processus d'hadronisation. i) Production d'une pair de
    quarks libres. ii) Échange de gluons entre les quarks. iii,iv) Les
    gluon émetent des paire $q\overline{q}$ à cause de la croissance
    rapide d'$\alpha_s$. v) Formation de hadrons lorsque l'énergie des
    quarks est rendu assez basse. Figure tirée de la
    référence~\cite{thomson_modern_2013}.}
  \label{fig:hadronisation}
\end{figure}

% \section{Validation expérimentale}
% \label{sec:ms:exp}

% \subsection{Mesures du secteur électrofaible}
% \label{sec:ms:exp:ewk}

% \subsection{Mesures du secteur chromodynamique}
% \label{sec:ms:exp:qcd}

% \subsection{Mesures du secteur Higgs}
% \label{sec:ms:exp:Higgs}

\section{Problèmes avec le Modèle Standard}
\label{sec:ms:problemes}

% intro
Malgré le succès phénoménal du Modèle Standard pour décrire la nature
à un niveau fondamental, la théorie contient quelques problèmes
importants qui motivent la recherche de théories sous-jacentes. \\

% Problème de la hiérarchie des masses
Un premier problème, celui de la \emph{hiérarchie des masses}, est
relié à la différence d'énergie importante entre l'échelle faible et
l'échelle de Planck, $M_P$, à laquelle les effets gravitationnels ne
peuvent plus être négligés~\footnote{$M_P = 2.4 \times 10^{18}$~GeV}. Puisque le MS ne décrit pas la gravité,
$M_P$ représente l'énergie à laquelle il faut absolument remplacer la
théorie présente par une plus fondamentale. Il se peut que le MS soit
remplacé à une énergie plus basse que $M_P$; cette échelle est notée $\Lambda_{UV}$.

% correction a la masse
Le problème de la hiérarchie des masses apparaît lorsqu'on considère
les corrections d'ordres supérieures à la masse du Higgs. Le Higgs peut
spontanément émettre une paire fermion-antifermion qui s'annihile
ensuite pour redonner un Higgs~(voir figure~\ref{fig:hloop}). Ce type de diagramme à boucle a pour
effet de modifier la masse du Higgs en ajoutant un terme~\cite{martin_supersymmetry_1997}.:

\begin{eqnarray}
\label{eq:higgs_fermion_corr}
\Delta m_H^2 \propto -\Lambda_{UV}^2
\end{eqnarray}

\begin{figure}
  \centering
  \includegraphics{higgs-loop.pdf}
  \caption{Correction à une boucle de la masse du Higgs par un fermion}
  \label{fig:hloop}
\end{figure}

% Mesure de m_H: fine tuning problem
La masse du Higgs a été mesuré en 2012 par les expériences ATLAS et
CMS et est de
$125.05 \pm 0.24$~GeV~\cite{atlas_collaboration_combined_2015}. Si
$\Lambda_{UV} \approx M_P$, alors la correction du diagramme à boucle
diverge de plusieurs ordres de grandeur par rapport la valeur mesurée
et Il faut alors soit ajuster une constante de proportionnalité à
environ $10^{-22}~\%$~\cite{giudice_naturally_20087} (ce qui est fait
dans le MS) ou remplacer le MS à un échelle $\Lambda_{UV} \ll M_P$ par
une théorie ayant une symétrie protégeant la masse du
Higgs~\cite{martin_supersymmetry_1997}. \\

% Matière sombre
En outre, certaines observations en astrophysique, par exemple dans la
mesure de la courbe de rotation des galaxies, présentent des anomalies
qui impliquent la présence d'amoncellements de particules massives
interagissant seulement par la force faible, appelés
\emph{WIMP}~\footnote{\emph{WIMP} $\equiv$ \emph{Weakly Interacting
    Massive Particle}}. Aucune particules dans le MS n'a les propriété
requises pour expliquer ces anomalies: c'est le problème de la
\emph{Matière Sombre}~\cite{bertone_particle_2005}. \\

% Unification
Un troisième problème théorique avec le MS est celui de
l'\emph{unification des couplages}. Il a été établie dans la
section~\ref{sec:ms:th:struct} que la constante de couplage forte
dépend de l'échelle d'énergie du processus considéré (voir
figure~\ref{fig:alpha_s}). Il en va de même pour les constantes de
couplages faibles et électromagnétiques qui, contrairement à
$\alpha_s$, augmentent en force avec l'échelle d'énergie. Si les
forces fortes, faibles et électriques sont fondamentalement des
manifestations d'une seule et même force dans une théorie
sous-jacente, alors il existe une échelle d'énergie d'unification,
$M_G$, à laquelle une seule constante, $\alpha_G$, définie la magnitude
des interactions. Les constantes de couplages à basse énergie doivent pouvoir
s'exprimer en fonction de $\alpha_G$. Or, une telle unification n'est
pas possible dans le Modèle Standard, et il faut donc considérer des
extensions à la théorie pour la réaliser~\cite{olive_gut_2014,thomson_modern_2013}.


%%% Local Variables:
%%% mode: latex
%%% TeX-master: "memoire"
%%% End:
