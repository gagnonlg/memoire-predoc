\singlespacing{}
\chapter{L'extension supersymétrique du Modèle Standard}
\label{sec:susy}
\doublespacing{}

Les problèmes mentionnées dans la section~\ref{sec:ms:problemes}
motivent la recherche de théories au-delà du Modèle Standard. Une de
ces extensions, la \emph{supersymétrie}, permettrait de régler de
façon élégante ces problèmes en introduisant seulement une nouvelle
symétrie. Une introduction à cette symétrie
particulière est présentée dans la section~\ref{sec:susy:th} et
l'extension supersymétrique minimale du Modèle Standard est décrite
dans la section~\ref{sec:susy:mssm}.

\section{Introduction à la supersymétrie}
\label{sec:susy:th}

Une supersymétrie est une symétrie qui associe un nouveau boson
scalaire à chaque fermion et un nouveau fermion à chaque boson du
modèle standard. Ces nouvelles particules sont les
\emph{superpartenaires} des particules du modèle standard. Si la supersymétrie est
une une symétrie exacte de la théorie, alors les superpartenaires
doivent avoir la même masse et les mêmes nombres quantiques (à part le
spin) que leurs partenaires respectifs. Cela signifie que la
supersymétrie est une symétrie brisée puisque sinon les superpartenaires
auraient déjà été observés. Une conséquence phénoménologique de cette
brisure de symétrie est que les masses des superpartenaires sont
différentes de celles de leurs partenaires. Le mécanisme de brisure
n'étant pas connu, les masses ne peuvent être prédites et entrent dans
le modèle comme paramètres~\cite{olive_susy1_2014}.

Étonnamment, cette idée toute simple est suffisante pour régler les
trois problèmes exposés dans la section~\ref{sec:ms:problemes}, ce qui
représente un fort argument en faveur de la supersymétrie. 

% Problème de l hiérarchie
D'une part, les particules supersymétrique vont rajouter des
corrections supplémentaires à la masse du Higgs qui annulent au moins
en partie les divergences dues aux particules du Modèle Standard. Par
exemple, les correction à la masse du higgs dues aux boucles de bosons
scalaires sont de la forme:
\begin{eqnarray}
  \Delta m_H^2 \propto 
  \left(
  \Lambda^2 - 2m^2_S ln(\Lambda/m_S)
  \right)
\end{eqnarray}

La partie quadratique de cette correction a un signe différent de la
formule~\ref{eq:higgs_fermion_corr} (correction dû à une boucle
fermionique). Si les constantes de proportionnalité étaient les mêmes,
les divergences quadratiques s'annulerait pour laisser seulement des
divergences logarithmiques et le modèle accommoderait beaucoup plus
naturellement la masse observée du Higgs. Cette relation n'est pas
présente entre les bosons et fermions du Modèle Standard, ce qui crée
le problème de la hiérarchie.  Or, il se trouve que les particules du
Modèle Standard \emph{et leurs superpartenaires du MSSM} ont
cette relation, signifiant que la supersymétrie pourrait régler le
problème de la hiérarchie si les corrections logarithmiques ne sont
pas trop élevées~\cite{martin_supersymmetry_1997}.

% Matière sombre
En outre, le problème de la matière sombre peut aussi être réglé par
le principe supersymétrique. En effet, plusieurs modèle
supersymétriques contiennent des particules ayant toutes les
caractéristiques des \emph{WIMP}
(c.f. section~\ref{sec:ms:problemes}), à savoir qu'elles sont
massives, stables et n'interagissent que par la force
faible~\cite{olive_susy1_2014}. Ce point sera couvert plus en détail
dans la section~\ref{sec:susy:R}.

% Unification
Finalement, l'ajout de nouvelles particules supersymétriques permet
d'unifier les couplages forts, faibles et électriques. En effet, la
présence de ces nouvelles particules modifie nécessairement la valeur
des couplages du Modèle Standard au-delà de l'échelle de brisure. Si
cette échelle est de l'ordre du TeV, alors les trois couplages
convergent vers une même valeur de $\alpha_G$ à $M_G \approx 10^{16}$
GeV~\cite{thomson_modern_2013} (voir figure~\ref{fig:unification}).
Comme discuté en section~\ref{sec:ms:problemes}, les couplages sont
alors unifiés.

\begin{figure}
  \centering
  \includegraphics{running_susy.pdf}
  \caption{Force des couplages dans le Modèle Standard (ligne
    pointillée) et dans le MSSM (lignes pleines) en fonction de
    l'échelle d'énergie ($Q$). Figure tirée de la
    réf.~\cite{martin_supersymmetry_1997}}
  \label{fig:unification}
\end{figure}

\section{Le Modèle Standard Minimalement Supersymétrique}
\label{sec:susy:mssm}

% intro
En soit, la supersymétrie n'est pas une théorie mais bien un principe
général de symétrie entre bosons et fermions qui pourrait faire partie
d'une théorie sous-jacente au Modèle Standard. Puisque la
supersymétrie requiert la présence de nouvelles particules, il faut
trouver l'ensemble minimal de particules supplémentaires qu'il faut
introduire au Modèle Standard pour avoir une théorie viable qui
possède cette symétrie. Puisque cette symétrie, si elle existe, est
nécessairement brisée en dessous d'une échelle d'au moins l'ordre du
TeV mais que le mécanisme de brisure n'est pas connu, il faut
incorporer au modèle une paramétrisation de ce mécanisme. Il en
résulte un modèle appelé \emph{Modèle Standard Minimalement
  Supersymétrique}, ou MSSM.

Au total, le MSSM compte 124 paramètres: 19 paramètres venant du
Modèle Standard et 105 nouveaux paramètres, la plupart étant les
masses des superpartenaires (elles ne sont pas prédites par la
théorie), des angles de mélange qui définissent les états physiques
ainsi que des phases quantifiant la violation CP additionnelle
introduite par le MSSM.

\subsection{Les superparticules du MSSM}
\label{sec:susy:mssm:sparticules}

\subsubsection{Superpartenaires des fermions}

Les fermions du Modèle Standard ont un spin de $\frac{1}{2}$. Il
existe donc pour chaque fermion deux états propres de
\emph{chiralité}, droit ($R$) et gauche ($L$), qui correspondent à la
projection du spin dans la direction du mouvement dans la limite
$E \gg m$ (hélicité). Il existe donc deux superpartenaires pour chaque
fermions, un par état de chiralité, sauf pour les neutrinos qui n'ont
qu'un état $L$~\cite{thomson_modern_2013}.

Ces \emph{sfermions} ne sont pas nécessairement des états propres de
masses, et les états physiques observables peuvent être des
combinaison linéaires des constituants de la paire associé à chaque
fermion. Dans la plupart des modèles le mélange entre les
superpartenaires des deux premières générations de quarks sont
négligeables et il est possible de les dénoter par la chiralité de leur
partenaire du modèle standard. Le mélange entre les sfermions de la
troisième génération ne peut en général pas être
négligé~\cite{aad_summary_2015}. 

La troisième génération de squark revêt un intérêt particulier
lorsqu'on considère le problème de la hiérarchie. En effet, ce sont
ces squarks qui, parmi les superpartenaires, occasionnent les
corrections à la masse du Higgs les plus
importantes~\cite{olive_susy2_2014}. Comme discuté dans la
section~\ref{sec:susy:th}, ces corrections sont proportionnelles à la
masse des particules et ne doivent pas être trop élevés pour que le
problème de la hiérarchie soit réglé dans le MSSM. Les masses des
squarks top (\emph{stop}, $\tilde{t}$) et bottom (\emph{sbottom},
$\tilde{b}$) doivent alors être de l'ordre du TeV et sont donc
activement recherchés au LHC~\cite{ATLAS-CONF-2015-067}.

\subsubsection{Superpartenaires des bosons}
%% gauginos
Comme vu en section~\ref{sec:ms:th:struct:ewk}, les bosons
électrofaibles observés sont des combinaisons linéaires des bosons du
Modèle Standard $W^{(1)}, W^{(2)}, W^{(3)}$ et $B$. De façon analogue,
les superpartenaires électrofaibles, les \emph{gauginos}, sont
associés à ces états fondamentaux et les états observés sont de façon
générale des combinaisons linéaires. Le mélange n'est cependant pas
prédit par la théorie et entre dans le modèle comme un ensemble de
paramètres~\cite{olive_susy1_2014}. Il y a deux types de gauginos:
ceux résultant du mélange de gauginos électriquement chargés, les
quatre \emph{charginos}, notés $\tilde{\chi}_{1,2}^\pm$ ainsi que ceux
résultant du mélange des gauginos neutres, les quatre
\emph{neutralinos}, notés $\tilde{\chi}_{1,2,3,4}^0$\footnote{Les
  indices sont ordonné selon la masse de l'état, en ordre
  croissant.}\cite{aad_summary_2015}.

%% higgsinos
Le MSSM contient plusieurs partenaires du Higgs, les
\emph{Higgsino}. Il y en a 5 au total (2 chargés, 3 neutres). C'est le
nombre minimal qu'il faut introduire dans la théorie pour éviter
l'apparition d'anomalies~\cite{olive_susy1_2014}.

%% gluinos
Le MSSM est complété par le \emph{gluino}, le superpartenaire du gluon
du Modèle Standard. Le gluino peut potentiellement apporter des
corrections importantes aux masses des squark top, et donc sa masse
doit aussi être de l'ordre du TeV pour que le problème de la
hiérarchie soit réglé dans le MSSM~\cite{ATLAS-CONF-2015-067}.

\subsection{La R-parité}
\label{sec:susy:R}

La force des couplages ne conservant pas les nombres baryoniques~($B$)
et leptoniques~($L$) dans les théorie au-delà du Modèle Standard est
fortement contrainte par les mesure de la stabilité du proton. En
effet, si $B$ et $L$ ne sont pas conservés, le proton pourrait par
exemple se désintégrer en un positron et un pion neutre. Pour
accommoder la présente limite inférieure mesurée du temps de vie du
proton de $2.1 \times 10^{29}$
ans~\cite{kamland_collaboration_search_2006}, il est nécessaire que de
tels couplages soient très faibles.  Dans le Modèle Standard, cette
conservation n'est pas un principe fondamental mais bien une
conséquence «accidentelle» de la structure du modèle: il n'y existe
tout simplement pas de termes possible violant la conservation de $B$
et $L$ qui soient renormalisables. Or, il est possible d'inclure des
interactions violant la conservation de $B$ et $L$ dans une extension
supersymétrique du modèle standard. Pour que la théorie soit viable,
il faut donc que ces couplages soit très faibles ou même nuls, ce qui
motive l'inclusion d'un principe de symétrie sous-jacent à la la
conservation de $B$ et $L$ dans le MSSM: la
\emph{R-parité}~\cite{martin_supersymmetry_1997}. Chaque particules
fondamentale se voit assigner une valeur de $R$-parité, qui dépend de
$B$, $L$ et du spin $S$ de la particule:
\begin{eqnarray}
  R = (-1)^{3(B - L) + 2S}
\end{eqnarray}

Toutes les particules du Modèle Standard ont une $R$-parité de $+1$,
tandis que leurs superpartenaires ont tous une $R$-parité de $-1$. Ce
nombre est multiplicativement conservé dans toutes les interactions du
MSSM, ce qui implique que tous les sommets d'interactions ont un
nombre pair de particules supersymétriques. Ceci a de grandes
conséquences phénoménologiques: les particules supersymétriques sont
toujours produites en nombre pairs lors de collisions de particules du
modèle standard, et toutes particule supersymétrique doit éventuellement se
désintégrer en états finaux contentant un nombre impair de particules
supersymétriques. Cette dernière implication est très importante:
cela signifie que la particule supersymétrique la plus légère est
stable~\cite{martin_supersymmetry_1997}. Cette particule est
différente selon la valeur des paramètres du modèles et est donc notée
$LSP$~\footnote{De l'anglais, \emph{Lightest Supersymmetric
    Particle}}. Si la LSP est le neutralino le plus léger, le
$\tilde{\chi}_1^0$, alors elle peut être un élément constituant la matière
sombre observée dans l'univers puisqu'elle a toute les caractéristique
d'une WIMP~\cite{olive_review_2014}.

% \subsection{Les paramètres du MSSM}
% \label{sec:susy:mssm:params}

% \subsection{Problèmes du MSSM}
% %* Problèmes avec le MSSM
% %% Pas une théorie complète (breaking?)
% %% Conserve pas nb leptonique
% %% trop de FCNC
% %% trop de violation CP
% %% c'est une bonne chose!

%%% Local Variables:
%%% mode: latex
%%% TeX-master: "memoire"
%%% End:
