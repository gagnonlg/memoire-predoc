\singlespacing{}
\chapter{L'extension supersymmétrique du Modèle Standard}
\label{sec:susy}
\doublespacing{}

Les problèmes mentionnées dans la section~\ref{sec:ms:problemes}
peuvent tous être réglés en introduisant une nouvelle symmétrie dans
le lagrangien du Modèle Standard: une \emph{supersymétrie}. Une
introduction à cette symétrie particulière est présentée dans la
section~\ref{sec:susy:th} et l'extension supersymétrique minimale du
Modèle Standard est décrite dans la section~\ref{sec:susy:mssm}. Cette
section se base majoritairement sur la
référence~\cite{olive_review_2014}, sauf indication contraire.

\section{Introduction à la supersymétrie}
\label{sec:susy:th}

Une supersymmétrie est une symmétrie qui associe un nouveau boson
fondamental à chaque fermion du Modèle Standard et vice-versa. Ces
nouvelle particules sont les \emph{superpartenaires} des particules du
MS. Si La supersymétrie est une une symétrie exacte de la théorie,
alors les superpartenaires doivent avoir la même masse et les mêmes
nombres quantiques (à part le spin) que leurs partenaires
respectifs. Cela signifie que la supersymmétrie est une symmétrie
brisée, sinon les superpartenaires auraient déjà été observés. Une
conséquence phénoménologique de cette brisure de symmétrie est que les
masses des superpartenaires sont différentes de celles de leurs
partenaires. Le mécanisme de brisure n'étant pas connu, les masses ne
peuvent être prédites et entrent dans le modèle comme paramètres.

Étonnament, cette idée toute simple est suffisante pour régler les
trois problèmes exposés dans la section~\ref{sec:ms:problemes}, ce qui
représente un fort argument en faveur de la supersymmétrie. 

% Problème de l hiérarchie
D'une part, les particules supersymmétrique vont rajouter des
corrections supplémentaires à la masse du Higgs mais avec le signe
contraire: les divergences quadratiques peuvent alors s'annuler pour
laisser seulement des divergences
logarithmiques~\cite{martin_supersymmetry_1997}. La masse observé du
Higgs devient alors beaucoup plus plausible puisqu'il n'y a plus
besoin de sur-ajuster une constante de proportionalité. (TODO: refaire
cette dernière phrase.) (TODO: rajouter corrections bosons dans
chap. 1)

% Matière sombre
% Unification

\section{Le Modèle Standard Minimalement Supersymmétrique}
\label{sec:susy:mssm}

% intro
En soit, la supersymétrie n'est pas une théorie mais bien une
symmétrie entre bosons et fermions qui pourrait faire partie d'une
théorie sous-jacente au Modèle Standard. Puisque la supersymmétrie
requiert la présence de nouvelles particules, il faut donc trouver
l'ensemble minimal de particules suplémmentaires qu'il faut introduire
au MS pour avoir une théorie viable où cette symétrie est présente,
mais qui est brisé à (au moins) en dessous de l'échelle du TeV (TODO
réécrire ca). Il en résulte une théorie qui incorpore le modèle
standard, toutes les super particules nécessaires ainsi qu'une
paramétrisation du modèle de brisure de symmétrie dont l'origine n'est
pas fixé à priori: le \emph{Modèle Standard Minimalement
  Supersymétrique}, ou MSSM. Le contenu additionel en particule sera
traité en section~\ref{sec:susy:mssm:sparticules} et la
paramétrisation sera discuté brièvement dans la
section~\ref{sec:susy:mssm:params}.

\subsection{Les superparticules du MSSM}
\label{sec:susy:mssm:sparticules}

\subsubsection{Superpartenaires des fermions}

Les fermions du Modèle Standard ont un spin de $\frac{1}{2}$. Il
existe donc pour chaque fermion deux états propres de chiralité, droit
($R$) et gauche ($L$), qui correspondent à la projection du spin dans
la direction du mouvement dans la limite $E \gg m$. Il existe donc
deux superpartenaires pour chaque fermions, un par état de chiralité,
sauf pour les neutrinos qui n'ont qu'un état $L$. 

Ces \emph{sfermions} ne sont pas nécessairement des états propres de
masses, et les états physiques observables peuvent être des
combinaison linéaires des constituants de la paire associé à chaque
fermion. Dans la plupart des modèles le mélange entre les
superpartenaires des deux premières générations de quarks (TODO:
générations chap. 1) sont négligeables et on peut dénoter les dénoter
par la chiralité de leur partenaire du MS. Le mélange entre les
sfermions de la troisième génération ne peut en général pas être
négligé~\cite{aad_summary_2015}. La troisième génération de squark
revêt un intérêt particulier, puisque leurs masses doivent être de
l'ordre du TeV pour que la supersymmétrie règle le problème de la
hiérarchie~\cite{olive_review_2014}.

\subsubsection{Superpartenaires des bosons}
%% gluinos
%% gauginos
%% higgsinos



\subsubsection{La R-parité}
%% R-parité
%% LSP

\subsection{Les paramètres du MSSM}
\label{sec:susy:mssm:params}

\subsection{Problèmes du MSSM}
%* Problèmes avec le MSSM
%% Pas une théorie complète (breaking?)
%% Conserve pas nb leptonique
%% trop de FCNC
%% trop de violation CP
%% c'est une bonne chose!

%%% Local Variables:
%%% mode: latex
%%% TeX-master: "memoire"
%%% End:
