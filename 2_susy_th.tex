\singlespacing{}
\chapter{L'extension supersymmétrique du Modèle Standard}
\label{sec:susy}
\doublespacing{}

Les problèmes mentionnées dans la section~\ref{sec:ms:problemes}
peuvent tous être réglés en introduisant une nouvelle symmétrie dans
le lagrangien du Modèle Standard: une \emph{supersymétrie}. Une
introduction à cette symétrie particulière est présentée dans la
section~\ref{sec:susy:th} et l'extension supersymétrique minimale du
Modèle Standard est décrite dans la section~\ref{sec:susy:mssm}.

\section{Introduction à la supersymétrie}
\label{sec:susy:th}

Une supersymmétrie est une symmétrie qui associe un nouveau boson
fondamental à chaque fermion du Modèle Standard et vice-versa. Ces
nouvelle particules sont les \emph{superpartenaires} des particules du
MS. Si La supersymétrie est une une symétrie exacte de la théorie,
alors les superpartenaires doivent avoir la même masse et les mêmes
nombres quantiques (à part le spin) que leurs partenaires
respectifs. Cela signifie que la supersymmétrie est une symmétrie
brisée, sinon les superpartenaires auraient déjà été observés. Une
conséquence phénoménologique de cette brisure de symmétrie est que les
masses des superpartenaires sont différentes de celles de leurs
partenaires. Le mécanisme de brisure n'étant pas connu, les masses ne
peuvent être prédites et entrent dans le modèle comme paramètres.

Étonnament, cette idée toute simple est suffisante pour régler les
trois problèmes exposés dans la section~\ref{sec:ms:problemes}, ce qui
représente un fort argument en faveur de la supersymmétrie. En effet,
les particules supersymmétrique vont rajouter des corrections
supplémentaires à la masse du Higgs mais avec le signe contraire: les
divergences quadratiques peuvent alors s'annuler pour laisser
seulement des divergences
logarithmiques~\cite{martin_supersymmetry_1997}. La masse observé du
Higgs devient alors beaucoup plus plausible puisqu'il n'y a plus
besoin de sur-ajuster une constante de proportionalité. (TODO: refaire
cette dernière phrase.) (TODO:  rajouter corrections bosons dans chap. 1)



\section{Le Modèle Standard Minimalement Supersymmétrique (MSSM)}
\label{sec:susy:mssm}

% \subsection{Description du MSSM}
% \label{sec:susy:mssm:descr}
 
% \subsection{Mécanismes de brisure de symmétrie}
% \label{sec:susy:mssm:break}

% \subsection{Le MSSM phénoménologique}
% \label{sec:susy:mssm:pmssm}



%%% Local Variables:
%%% mode: latex
%%% TeX-master: "memoire"
%%% End:
