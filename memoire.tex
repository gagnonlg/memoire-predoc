\documentclass[12pt,canadien]{report}

\usepackage{babel}
\usepackage[utf8]{inputenc}
\usepackage[T1]{fontenc}
\usepackage{fullpage}
\usepackage{setspace}

\usepackage{blindtext}

\title{Recherce de partenaires supersymmétriques des gluons se
  désintégrant en quarks top de hautes énergies à l'expérience ATLAS}

\author{Louis-Guillaume Gagnon}

\begin{document}

\maketitle{}
\doublespacing

\tableofcontents{}

\chapter{Le Modèle Standard et ses limitations}
\label{sec:ms}

\section{Survol théorique}
\label{sec:ms:th}

\section{Validation expérimentale}
\label{sec:ms:exp}

\section{Problèmes avec le Modèle Standard}
\label{sec:ms:problemes}

\singlespacing{}
\chapter{L'extension supersymmétrique du Modèle Standard}
\label{sec:susy}
\doublespacing{}

\section{Survol théorique}
\label{sec:susy:th}

\section{Contraintes expérimentales}
\label{sec:susy:contraintes}

\singlespacing{}
\chapter{Le Grand Collisionneur de Hadrons (LHC) et le détecteur
  ATLAS}
\label{sec:lhc_atlas}
\doublespacing{}

\section{Le LHC}
\label{sec:lhc_atlas:lhc}

\section{Le détecteur ATLAS}
\label{sec:lhc_atlas:atlas}

\singlespacing{}
\chapter{La reconstruction de quarks top à haute énergie à ATLAS}
\label{sec:top}
\doublespacing{}

\section{Les quarks tops à haute impulsion transverse}
\label{sec:top:boosted}

\section{Les méthodes de sous-structure}
\label{sec:top:sous_structure}

\section{Identification par apprentissage machine}
\label{sec:top:ml}

% le titre donné dans la lettre terminait par: «avec un focus sur la
% recherche du partenaire supersymmétrique du gluon, le gluino, se
% désintégrant en quarks top »
\singlespacing{}
\chapter{La recherce de la Supersymmétrie à ATLAS}
\label{susy_atlas}
\doublespacing{}

\section{Le modèle Gtt}
\label{sec:susy_atlas:gtt}

\end{document}
