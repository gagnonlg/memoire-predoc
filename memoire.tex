\documentclass[12pt,canadien]{article}

\usepackage{babel}
\usepackage[utf8]{inputenc}
\usepackage[T1]{fontenc}
\usepackage{fullpage}
\usepackage{setspace}

%\setcounter{secnumdepth}{5}
%\setcounter{tocdepth}{5}

\usepackage{blindtext}

\title{\textbf{Recherce de partenaires supersymmétriques des gluons se
  désintégrant en quarks top de hautes énergies à l'expérience ATLAS}}

\author{Louis-Guillaume Gagnon \\ Candidat au doctorat en physique \\ Mémoire présenté dans le cadre de l'examen général de synthèse}

\begin{document}


%titlepage
\thispagestyle{empty}
\begin{center}
\begin{minipage}{0.75\linewidth}
    \centering
%University logo
    %\includegraphics[width=0.3\linewidth]{logo.pdf}
    %\rule{0.4\linewidth}{0.15\linewidth}\par
    \vspace{3cm}
%Thesis title
    {\bf {\Large Recherce de partenaires supersymmétriques des gluons se
  désintégrant en quarks top de hautes énergies à l'expérience ATLAS \\ }}
    \vspace{3cm}
%Author's name
    {{\Large Louis-Guillaume Gagnon}\\Candidat au doctorat\\ Département de physique\\ Université de Montréal\par}
    \vspace{3cm}
%Degree
    {\Large Mémoire présenté dans le cadre de l'examen général de synthèse\\}
    \vspace{4cm}
%Date
    {\Large \today}
\end{minipage}
\end{center}
\clearpage


%\maketitle
%\clearpage
\tableofcontents{}
\clearpage
\doublespacing

\section{Le Modèle Standard et ses limitations}
\label{sec:ms}

\blindtext

\subsection{Survol théorique}
\label{sec:ms:th}

\blindtext[5]

\subsection{Validation expérimentale}
\label{sec:ms:exp}

\subsection{Problèmes avec le Modèle Standard}
\label{sec:ms:problemes}

\singlespacing{}
\section{L'extension supersymmétrique du Modèle Standard}
\label{sec:susy}
\doublespacing{}

\subsection{Survol théorique}
\label{sec:susy:th}

\subsubsection{le MSSSM}
\label{sec:susy:th:mssm}

\subsection{Contraintes expérimentales}
\label{sec:susy:contraintes}

\subsubsection{le pMSSM}
\label{sec:susy:contraintes:pmssm}



\singlespacing{}
\section{Le Grand Collisionneur de Hadrons (LHC) et le détecteur
  ATLAS}
\label{sec:lhc_atlas}
\doublespacing{}

\subsection{Le LHC}
\label{sec:lhc_atlas:lhc}

\subsection{Le détecteur ATLAS}
\label{sec:lhc_atlas:atlas}

\subsubsection{Le détecteur interne}
\label{sec:lhc_atlas:atlas:indet}

\subsubsection{Les calorimètres}
\label{sec:lhc_atlas:atlas:calo}

\subsubsection{Le spectromètre à muon}
\label{sec:lhc_atlas:atlas:mu}

\subsubsection{Aquisition des données}
\label{sec:lhc_atlas:atlas:daq}

% \subsubsubsection{Les déclancheurs}
% \label{sec:lhc_atlas:atlas:daq:trig}

\singlespacing{}
\section{La reconstruction de quarks top à haute énergie à ATLAS}
\label{sec:top}
\doublespacing{}

\subsection{Les quarks tops à haute impulsion transverse}
\label{sec:top:boosted}

\subsection{Les méthodes de sous-structure}
\label{sec:top:sous_structure}

\subsubsection{Variables de sous-structure}
\label{sec:top:sous_structure:variables}

\subsubsection{Shower Deconstruction}
\label{sec:top:sous_structure:shower}

\subsubsection{HepTopTagger}
\label{sec:top:sous_structure:hep}

\subsubsection{Comparaison des différentes méthodes}
\label{sec:top:sous_structure:comp}


\subsection{Identification par apprentissage machine}
\label{sec:top:ml}

% le titre donné dans la lettre terminait par: «avec un focus sur la
% recherche du partenaire supersymmétrique du gluon, le gluino, se
% désintégrant en quarks top »
\singlespacing{}
\section{La recherce de la Supersymmétrie à ATLAS}
\label{susy_atlas}
\doublespacing{}

\subsection{Les modèles simplifiés}
\label{sec:susy_atlas:modele_simple}

\subsection{Désintégrations des gluinos en tops: Le modèle Gtt}
\label{sec:susy_atlas:gtt}

\subsection{Recherches par apprentissage machine}
\label{sec:susy_atlas:ml}

\section{Conclusion}
\label{sec:conclusion}




\end{document}
